\documentclass[dvipdfmx,cjk,xcolor=dvipsnames,envcountsect,notheorems,12pt]{beamer}
% * 16:9 のスライドを作るときは、aspectratio=169 を documentclass のオプションに追加する
% * 印刷用の配布資料を作るときは handout を documentclass のオプションに追加する
%   (overlay が全て一つのスライドに出力される)

\usepackage{pxjahyper}% しおりの文字化け対策 (なくても良い)
\usepackage{amsmath,amssymb,amsfonts,amsthm,ascmac,cases,bm,pifont}
\usepackage{graphicx}
\usepackage{bussproofs}
\usepackage{url}
\usepackage{etex}

% スライドのテーマ
\usetheme{sumiilab}
% ベースになる色を指定できる
%\usecolortheme[named=Magenta]{structure}
% 数式の文字が細くて見難い時は serif の代わりに bold にしましょう
%\mathversion{bold}

%% ===============================================
%% スライドの表紙および PDF に表示される情報
%% ===============================================

%% 発表会の名前とか(省略可)
\session{日本ソフトウェア科学会第33回大会}
%% スライドのタイトル
\title{無限の入出力を行う関数型プログラムのK正規化の形式的検証}
%% 必要ならば、サブタイトルも
%\subtitle{}
%% 発表者のお名前
\author{水野雅之 住井英二郎}
%% 発表者の所属([] 内は短い名前)
% \institute[東北大学 住井・松田研]{工学部 情報知能システム総合学科\\住井・松田研究室}% 学部生
\institute[東北大学 住井・松田研]{東北大学 大学院情報科学研究科}% 院生
%% 発表する日
\date{2016年9月9日}

%% ===============================================
%% 自動挿入される目次ページの設定(削除しても可)
%% ===============================================

%% section の先頭に自動挿入される目次ページ(削除すると、表示されなくなる)
\AtBeginSection[]{
\begin{frame}
  \frametitle{アウトライン}
  \tableofcontents[sectionstyle=show/shaded,subsectionstyle=show/show/hide]
\end{frame}}
%% subsection の先頭に自動挿入される目次ページ(削除すると、表示されなくなる)
\AtBeginSubsection[]{
\begin{frame}
  \frametitle{アウトライン}
  \tableofcontents[sectionstyle=show/shaded,subsectionstyle=show/shaded/hide]
\end{frame}}

%% 現在の section 以外を非表示にする場合は以下のようにする

%% \AtBeginSection[]{
%% \begin{frame}
%%   \frametitle{アウトライン}
%%   \tableofcontents[sectionstyle=show/hide,subsectionstyle=show/show/hide]
%% \end{frame}}
%% \AtBeginSubsection[]{
%% \begin{frame}
%%   \frametitle{アウトライン}
%%   \tableofcontents[sectionstyle=show/hide,subsectionstyle=show/shaded/hide]
%% \end{frame}}

%% ===============================================
%% 定理環境の設定
%% ===============================================

\setbeamertemplate{theorems}[numbered]% 定理環境に番号を付ける
\theoremstyle{definition}
\newtheorem{definition}{定義}
\newtheorem{axiom}{公理}
\newtheorem{theorem}{定理}
\newtheorem{lemma}{補題}
\newtheorem{corollary}{系}
\newtheorem{proposition}{命題}

%% ===============================================
%% ソースコードの設定
%% ===============================================

\usepackage{listings,jlisting}
%\usepackage[scale=0.9]{DejaVuSansMono}

\definecolor{DarkGreen}{rgb}{0,0.5,0}
% プログラミング言語と表示するフォント等の設定
\lstset{
  language={[Objective]Caml},% プログラミング言語
  basicstyle={\ttfamily\small},% ソースコードのテキストのスタイル
  keywordstyle={\bfseries},% 予約語等のキーワードのスタイル
  commentstyle={},% コメントのスタイル
  stringstyle={},% 文字列のスタイル
  frame=trlb,% ソースコードの枠線の設定 (none だと非表示)
  numbers=none,% 行番号の表示 (left だと左に表示)
  numberstyle={},% 行番号のスタイル
  xleftmargin=5pt,% 左余白
  xrightmargin=5pt,% 右余白
  keepspaces=true,% 空白を表示する
  mathescape,% $ で囲った部分を数式として表示する ($ がソースコード中で使えなくなるので注意)
  % 手動強調表示の設定
  moredelim=[is][\itshape]{@/}{/@},
  moredelim=[is][\color{red}]{@r\{}{\}@},
  moredelim=[is][\color{blue}]{@b\{}{\}@},
  moredelim=[is][\color{DarkGreen}]{@g\{}{\}@},
}

\newcommand{\keyword}[1]{\mathbf{#1}}
\newcommand{\LET}{\keyword{let}}
\newcommand{\IF}{\keyword{if}}
\newcommand{\THEN}{\keyword{then}}
\newcommand{\ELSE}{\keyword{else}}
\newcommand{\FIX}{\keyword{fix}}
\newcommand{\CREATE}{\keyword{create}}
\newcommand{\AND}{\keyword{and}}
\newcommand{\IN}{\keyword{in}}
\newcommand{\TRUE}{\keyword{true}}
\newcommand{\WHILE}{\keyword{while}}
\newcommand{\DO}{\keyword{do}}
\newcommand{\DONE}{\keyword{done}}

%% ===============================================
%% 本文
%% ===============================================
\begin{document}
\frame[plain]{\titlepage}% タイトルページ

\begin{frame}
	\frametitle{研究の貢献}
	\LARGE 入出力を含む関数型言語における\\
	K正規化を検証
	\begin{itemize}
		\item 無限に入出力が続く場合も検証
			\begin{itemize}
				\item 余帰納的大ステップ意味論
			\end{itemize}
		\item 証明を簡潔に
			\begin{itemize}
				\item de Buijnインデックス
			\end{itemize}
	\end{itemize}
\end{frame}

\section*{アウトライン}

% 目次を表示させる(section を表示し、subsection は隠す)
\begin{frame}
  \frametitle{アウトライン}
  \tableofcontents[sectionstyle=show,subsectionstyle=hide]
\end{frame}

\section{研究背景}

\begin{frame}
	\frametitle{コンパイラの形式的検証の意義}
	\LARGE コンパイラのバグ
	\begin{itemize}
		\item 生成されるコードに影響が及ぶ
		\item デバッグが困難
	\end{itemize}

	\vfill

	信頼性の高いソフトウェアには\\信頼性の高いコンパイラが必要
\end{frame}

\begin{frame}
	\frametitle{関連研究(1/2)}
	\LARGE CompCert [Leroyら 2006]
	\begin{itemize}
		\item Cコンパイラの正当性の検証
			\begin{itemize}
				\item 束縛の操作は限定的
			\end{itemize}
	\end{itemize}

	Lambda Tamer [Chlipala 2010]
	\begin{itemize}
		\item 非純粋関数型言語処理系の正当性の検証
			\begin{itemize}
				\item 入出力はサポートしない
			\end{itemize}
	\end{itemize}

	\vfill

	入出力がある関数型言語は少ない
\end{frame}

\begin{frame}
	\frametitle{関連研究(2/2)}
	\LARGE
	独立に行われた同様の研究

	\vfill

	CakeML [Tanら 2016]
	\begin{itemize}
		\item ソース言語は名前で束縛を表現
		\item 中間言語はde Bruijnインデックス
		\item クロックを用いた大ステップ意味論
	\end{itemize}

	Pilsner [Neisら 2015]
	\begin{itemize}
		\item 名前で束縛を表現
		\item 小ステップ意味論
		\item 論理関係
	\end{itemize}
\end{frame}

\section{対象言語}

\begin{frame}
	\frametitle{対象言語}
	\Large
	\begin{itemize}
		\item \alert<1>{高階関数}
		\item \alert<2>{複合的な値(組)}
		\item \alert<3>{外部関数呼び出し(入出力)}
	\end{itemize}
	{\normalsize
	\[ \begin{array}{lcl}
		e & ::= & x	\\
				& & \vdots \\
				& | & \alert<1>{\FIX~f~x.~e} \\
				& | & \alert<1,3>{e_1~e_2} \\
				& | & \alert<2>{(e_1,~e_2)} \\
				& | & \alert<2>{\pi_i~e} \\
			v & ::= & c \\
				& | & \alert<3>{x} \\
				& | & \alert<1>{\FIX~f~x.~e} \\
				& | & \alert<2>{(v_1,~v_2)} \\
	\end{array} \]}
	\pause
	\pause
\end{frame}

\begin{frame}
	\frametitle{外部関数呼び出し}
	\LARGE \alert<2>{自由変数}への関数呼び出し

	\vfill

	\begin{center}
		\Large
		$ \alert<2>{\texttt{print\_int}}~42 $
	\end{center}

	\vfill

	外部関数呼び出しとみなす

	\pause
\end{frame}

\section{K正規化}

\begin{frame}
	\frametitle{K正規化}
	\LARGE
	全ての部分式に名前を付ける

	\begin{columns}
		\begin{column}{0.5\textwidth}
			\begin{center}
				$a+b*c+d$
			\end{center}
		\end{column}
		\begin{column}{0.5\textwidth}
			\begin{center}
				\[
					\begin{array}{l}
						\alert<2>{\LET~x =} b * c~\alert<2>{\IN} \\
						\alert<2>{\LET~y =} a + x~\alert<2>{\IN} \\
						y + d
					\end{array}
				\]
			\end{center}
		\end{column}
	\end{columns}

	\vfill

	\alert<2>{束縛に関する操作}
	\pause
\end{frame}

\section{束縛の表現}

\begin{frame}
	\frametitle{名前による表現の問題点}
	\LARGE $\alpha$等価性の議論が面倒
	\[ \lambda x.\lambda y.~x \simeq \lambda a.\lambda b.~a \]

	\vfill

	freshな名前が必要になる
	\begin{itemize}
		\item 束縛の関係を乱さないよう変数名を選ぶ
	\end{itemize}
	\[
		\begin{array}{lcl}
			[x \mapsto z](\lambda z.~x) & \simeq & \lambda z'.~z \\
																	& \not \simeq & \lambda z.~z
		\end{array}
	\]
\end{frame}

\begin{frame}
	\frametitle{de Bruijnインデックス}
	\LARGE
	何番目の束縛かで変数の出現を表現
	\begin{itemize}
		\item 内側から外側へ数える
	\end{itemize}
	\begin{columns}
		\begin{column}{0.5\textwidth}
			\[ \lambda x.~x~(\lambda y.~x~y) \]
		\end{column}
		\begin{column}{0.5\textwidth}
			\[ \lambda.~0~(\lambda.~1~0) \]
		\end{column}
	\end{columns}

	\vfill

	$\alpha$等価な式は構文的に等価
	\begin{itemize}
		\item 名前のfreshnessを保証
	\end{itemize}
	\begin{columns}
		\begin{column}{0.5\textwidth}
			\[ \lambda x.\lambda y.~x \]
			\[ \lambda a.\lambda b.~a \]
		\end{column}
		\begin{column}{0.5\textwidth}
			\[ \lambda.\lambda.~1 \]
		\end{column}
	\end{columns}
\end{frame}

\begin{frame}
	\frametitle{シフト}
	\LARGE 自由変数のインデックスをずらす
	\[\uparrow^d t \]

	束縛に関する操作

	\vfill

	{\large \begin{columns}
		\begin{column}{0.5\textwidth}
			\centering
			$\begin{array}{l}
				\lambda z.~(\lambda x.\lambda y.~x)~z \\
				\rightarrow \lambda z.\lambda y.~z
			\end{array}$
		\end{column}
		\begin{column}{0.5\textwidth}
			\centering
			$\begin{array}{l}
				\lambda.~(\lambda.\lambda.~1)~0\\
				\rightarrow \lambda.\lambda.~\only<2>{\alert{1}}\only<1>{\alert{\uparrow^1 0}}
			\end{array}$
		\end{column}
	\end{columns}}
	\pause
\end{frame}

\begin{frame}[fragile]
	\frametitle{K正規化の実装}
	\LARGE シフトを用いる
	{\large \[\begin{array}{rcl}
		K(x) & \equiv & x \\
		K(\FIX.~e) & \equiv & \FIX.~K(e) \\
		K(e_1~e_2) & \equiv & \LET~ K(e_1)~\IN~\LET~ \uparrow^1 K(e_2)~\IN~1~0 \\
	\end{array}\]}
\end{frame}

\begin{frame}
	\frametitle{本研究での自由変数の表現}
	\LARGE
	自由変数は名前で表現

	{\Large \[(\FIX~f~x.~\texttt{print\_int}~x)~42\]

	\vfill

	\[\texttt{App}~(\texttt{Fix}~(\texttt{App}~(\texttt{ExtVar}~v)~(\texttt{Var}~1)))~(\texttt{Int}~42)\]}
	{\large \begin{center}
		($v$は$\texttt{print\_int}$を表す自然数)
	\end{center}}
\end{frame}

\begin{frame}
	\frametitle{局所名前無し表現}
	\LARGE 自由変数は名前で表現
	\begin{itemize}
		\item シフトの代わりにopenとclose
		\item locally closed
	\end{itemize}

	\only<1>{\begin{center}
		$\lambda.~(\lambda.\lambda.~1)~0 \rightarrow \lambda.\lambda.~1$
	\end{center}}
	\only<2>{\begin{prooftree}
		\AxiomC{$\alert{((\lambda.\lambda.~1)~0)^z} \rightarrow \alert{(\lambda.~1)^z}$}
		\RightLabel{{\large \textsc{E-Abs}}}
		\UnaryInfC{$\lambda.~(\lambda.\lambda.~1)~0 \rightarrow \lambda.\lambda.~1 $}
	\end{prooftree}}
	\only<3>{\begin{prooftree}
		\AxiomC{$\alert{(\lambda.\lambda.~1)~z} \rightarrow \alert{\lambda.~z}$}
		\RightLabel{{\large \textsc{E-Abs}}}
		\UnaryInfC{$\lambda.~(\lambda.\lambda.~1)~0 \rightarrow \lambda.\lambda.~1 $}
	\end{prooftree}}
	\only<4>{\begin{prooftree}
		\AxiomC{}
		\RightLabel{{\large \textsc{E-AppAbs}}}
		\UnaryInfC{$(\lambda.\lambda.~1)~z \rightarrow \alert{(\lambda.~1)^z} $}
		\RightLabel{{\large \textsc{E-Abs}}}
		\UnaryInfC{$\lambda.~(\lambda.\lambda.~1)~0 \rightarrow \lambda.\lambda.~1 $}
	\end{prooftree}}
\end{frame}

\begin{frame}
	\frametitle{パラメトリック高階抽象構文(PHOAS)}
	\LARGE
	ホスト言語の束縛を流用
	\[\texttt{Abs}~(\texttt{fun}~x \Rightarrow \texttt{Var}~x)\]

	\begin{itemize}
		\item 名前の付け替えや代入をメタ言語に任せられる
		\item メタ言語の性質に関する証明が必要になる
	\end{itemize}
\end{frame}

\section{意味論の定義}

\begin{frame}
	\frametitle{大ステップ意味論}
	\LARGE
	\begin{center}
	比較的単純な\\プログラム変換の検証に適する

	\vfill

	が

	\vfill

	無限ループとエラーの区別が困難
	\end{center}
\end{frame}

\begin{frame}
	\frametitle{例:型無しラムダ計算}
	\Large
	構文
	\begin{columns}
		\begin{column}{0.5\textwidth}
			\[
				\begin{array}{lcl}
					t & ::= & b \\
					  & | & x \\
						& | & \lambda x.~t \\
						& | & t~t
				\end{array}
			\]
		\end{column}
		\begin{column}{0.5\textwidth}
			\[
				\begin{array}{lcl}
					v & ::= & b \\
						& | & \lambda x.~t \\
				\end{array}
			\]
		\end{column}
	\end{columns}

	\vfill

	意味論
	\begin{columns}
		\begin{column}{0.5\textwidth}
			\[\overline{b \Downarrow b}\]
		\end{column}
		\begin{column}{0.5\textwidth}
			\[\overline{\lambda x.~t \Downarrow \lambda x.~t}\]
		\end{column}
	\end{columns}

	\begin{prooftree}
		\AxiomC{$t_1 \Downarrow \lambda x.~t$}
		\AxiomC{$t_2 \Downarrow v_2$}
		\AxiomC{$[x \mapsto v_2]t \Downarrow v$}
		\TrinaryInfC{$t_1~t_2 \Downarrow v$}
	\end{prooftree}
\end{frame}

\begin{frame}
	\frametitle{大ステップ意味論の問題点}
	\begin{columns}
		\begin{column}{0.4\textwidth}
			{\LARGE エラー}
			{\Large \[ \TRUE~\TRUE \not \Downarrow v \]}
			適用できる規則が無い
		\end{column}
		\begin{column}{0.6\textwidth}
			{\LARGE 無限ループ}
			{\Large \[ ~(\lambda x.xx)(\lambda x.xx) \not \Downarrow v \]}
			有限回の規則適用で導出できない
		\end{column}
	\end{columns}

	\vfill

	\begin{center}
		{\LARGE 区別できない}
	\end{center}
\end{frame}

\begin{frame}
	\frametitle{余帰納的大ステップ意味論(1/2)}
	\LARGE
	余帰納的に定義 [Leroy, Grall 2009]
	\begin{columns}
		\begin{column}{0.3\textwidth}
			\begin{prooftree}
				\AxiomC{$t_1 \Uparrow$}
				\doubleLine
				\UnaryInfC{$t_1~t_2 \Uparrow$}
			\end{prooftree}
		\end{column}
		\begin{column}{0.7\textwidth}
			\begin{prooftree}
				\AxiomC{$t_1 \Downarrow v_1$}
				\AxiomC{$t_2 \Uparrow$}
				\doubleLine
				\BinaryInfC{$t_1~t_2 \Uparrow$}
			\end{prooftree}
		\end{column}
	\end{columns}

	\vfill

	\begin{prooftree}
		\AxiomC{$t_1 \Downarrow \lambda x.~t$}
		\AxiomC{$t_2 \Downarrow v_2$}
		\AxiomC{$[x \mapsto v_2]t \Uparrow$}
		\doubleLine
		\TrinaryInfC{$t_1~t_2 \Uparrow$}
	\end{prooftree}
\end{frame}

\begin{frame}
	\frametitle{余帰納的大ステップ意味論(2/2)}
	\LARGE
	\begin{columns}
		\begin{column}{0.4\textwidth}
			エラー
			\[ \TRUE~\TRUE \not \Uparrow \]
			{\normalsize 適用できる規則がない}
		\end{column}
		\begin{column}{0.6\textwidth}
			無限ループ
			\[ (\lambda x.x x)(\lambda x.x x) \Uparrow \]
			{\normalsize 無限回の規則適用を許す}
		\end{column}
	\end{columns}

	\vfill

	\begin{center}
		区別できる
	\end{center}
\end{frame}

\begin{frame}
	\frametitle{参考:入出力を含む言語への拡張}
	\Large どのような入出力を行ったかを表すラベルの列を付与
	{\normalsize \[
			\begin{array}{l}
				\keyword{print\_endline}~\mathit{"hoge"};~\keyword{read\_line}~()\\
				\Downarrow "fuga"~/~(\keyword{print\_endline}~\mathit{"hoge"},~\keyword{read\_line}~()="fuga")
			\end{array}
	\]}

	ストリームを用いれば無限回の入出力も
	{\normalsize \[
		\begin{array}{l}
			(\WHILE~\TRUE~\DO~\keyword{print\_endline}~\mathit{"hoge"}\DONE) \\
			\Uparrow /~(\keyword{print\_endline}~\mathit{"hoge"}, \keyword{print\_endline}~\mathit{"hoge"}, \cdots)
		\end{array}
	\]}
\end{frame}


\section{正当性の検証}

\begin{frame}
	\frametitle{期待される性質(1/2)}
	{\LARGE K正規化後の項を評価してみる}

	\vfill

	\Large 単純な値では一致
	{\normalsize \begin{columns}
		\begin{column}{0.4\textwidth}
			\[ 1+2\Downarrow 3 \]
		\end{column}
		\begin{column}{0.4\textwidth}
			\[ 
				\begin{array}{l}
					\LET~a = 1~\IN \\
					\LET~b = 2~\IN \\
					a + b\Downarrow 3
				\end{array}
			\]
		\end{column}
	\end{columns}}

	\vfill

	関数抽象の中は評価されない
	{\normalsize \begin{columns}
		\begin{column}{0.4\textwidth}
			\[
				\begin{array}{l}
					\lambda x.\lambda y.\lambda z.~x+y+z\\
					\Downarrow \lambda x.\lambda y.\lambda z.~x+y+z
				\end{array}
			\]
		\end{column}
		\begin{column}{0.4\textwidth}
			\[ 
				\begin{array}{l}
					\lambda x.\lambda y.\lambda z.\\
					\LET~a = x+y~\IN \\
					a + z \\
					\Downarrow \lambda x.\lambda y.\lambda z.\\
					\LET~a = x+y~\IN \\
					a + z \\
				\end{array}
			\]
		\end{column}
	\end{columns}}
\end{frame}

\begin{frame}
	\frametitle{期待される性質(2/2)}
	{\Large
	\begin{theorem}
		項tが値vに評価される場合、項tをK正規化した結果K(t)は値vをK正規化した結果K(v)に評価される
	\end{theorem}
	\begin{theorem}
		項tの評価が停止しない場合、項tをK正規化した結果K(t)の評価は停止しない
	\end{theorem}}
	\Large 今回検証した言語の範囲では評価は決定的なため、逆も成り立つ

\end{frame}


\section{結論}

\begin{frame}
	\frametitle{結論}
	\Large K正規化をCoqで検証できた
	\begin{itemize}
		\item de Buijnインデックス、\\余帰納的大ステップ意味論の採用で証明が簡潔に
		\item 証明自動化による再利用性の高い証明
	\end{itemize}
\end{frame}

\begin{frame}
	\frametitle{今後の課題}
	\Large さらに言語を拡張し、MinCamlと同等に
	\begin{itemize}
		\item 組や複数引数の関数
		\item 配列
		\item 外部関数呼び出し
	\end{itemize}

	\vfill

	K正規化以外の検証も
\end{frame}

\end{document}
