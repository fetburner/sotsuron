\documentclass[t,dvipdfmx,cjk,xcolor=dvipsnames,envcountsect,notheorems,12pt]{beamer}
% * 16:9 のスライドを作るときは、aspectratio=169 を documentclass のオプションに追加する
% * 印刷用の配布資料を作るときは handout を documentclass のオプションに追加する
%   (overlay が全て一つのスライドに出力される)

\usepackage{pxjahyper}% しおりの文字化け対策 (なくても良い)
\usepackage{amsmath,amssymb,amsfonts,amsthm,ascmac,cases,bm,pifont}
\usepackage{graphicx}
\usepackage{bussproofs}
\usepackage{url}
\usepackage{etex}
\usepackage{fancybox}
\usepackage{newtxmath}
\usepackage{tikz}
\usepackage{pxpgfmark}
\usetikzlibrary{positioning}
\usetikzlibrary{arrows}
\usetikzlibrary{shapes.callouts}

% スライドのテーマ
\usetheme{sumiilab}
% ベースになる色を指定できる
%\usecolortheme[named=Magenta]{structure}
% 数式の文字が細くて見難い時は serif の代わりに bold にしましょう
%\mathversion{bold}

%% ===============================================
%% スライドの表紙および PDF に表示される情報
%% ===============================================

%% 発表会の名前とか(省略可)
\session{日本ソフトウェア科学会第33回大会}
%% スライドのタイトル
\title{\alert<2>{無限の入出力}を行う\\\alert<2>{関数型プログラム}のK正規化の\\形式的検証}
%% 必要ならば、サブタイトルも
%\subtitle{}
%% 発表者のお名前
\author{水野雅之 住井英二郎}
%% 発表者の所属([] 内は短い名前)
% \institute[東北大学 住井・松田研]{工学部 情報知能システム総合学科\\住井・松田研究室}% 学部生
\institute[東北大学 住井・松田研]{東北大学 大学院情報科学研究科}% 院生
%% 発表する日
\date{2016年9月9日}

%% ===============================================
%% 自動挿入される目次ページの設定(削除しても可)
%% ===============================================

%% section の先頭に自動挿入される目次ページ(削除すると、表示されなくなる)
\AtBeginSection[]{
\begin{frame}
  \frametitle{アウトライン}
  \tableofcontents[sectionstyle=show/shaded,subsectionstyle=show/hide/hide]
\end{frame}}
%% subsection の先頭に自動挿入される目次ページ(削除すると、表示されなくなる)
\AtBeginSubsection[]{
\begin{frame}
  \frametitle{アウトライン}
  \tableofcontents[sectionstyle=show/shaded,subsectionstyle=show/shaded/hide]
\end{frame}}

%% 現在の section 以外を非表示にする場合は以下のようにする

%% \AtBeginSection[]{
%% \begin{frame}
%%   \frametitle{アウトライン}
%%   \tableofcontents[sectionstyle=show/hide,subsectionstyle=show/show/hide]
%% \end{frame}}
%% \AtBeginSubsection[]{
%% \begin{frame}
%%   \frametitle{アウトライン}
%%   \tableofcontents[sectionstyle=show/hide,subsectionstyle=show/shaded/hide]
%% \end{frame}}

%% ===============================================
%% 定理環境の設定
%% ===============================================

\setbeamertemplate{theorems}[numbered]% 定理環境に番号を付ける
\theoremstyle{definition}
\newtheorem{definition}{定義}
\newtheorem{axiom}{公理}
\newtheorem{theorem}{定理}
\newtheorem{lemma}{補題}
\newtheorem{corollary}{系}
\newtheorem{proposition}{命題}
\newtheorem{hypothesis}{仮説}

%% ===============================================
%% ソースコードの設定
%% ===============================================

\usepackage{listings,jlisting}
%\usepackage[scale=0.9]{DejaVuSansMono}

\definecolor{DarkGreen}{rgb}{0,0.5,0}
% プログラミング言語と表示するフォント等の設定
\lstset{
  language={[Objective]Caml},% プログラミング言語
  basicstyle={\ttfamily\small},% ソースコードのテキストのスタイル
  keywordstyle={\bfseries},% 予約語等のキーワードのスタイル
  commentstyle={},% コメントのスタイル
  stringstyle={},% 文字列のスタイル
  frame=trlb,% ソースコードの枠線の設定 (none だと非表示)
  numbers=none,% 行番号の表示 (left だと左に表示)
  numberstyle={},% 行番号のスタイル
  xleftmargin=5pt,% 左余白
  xrightmargin=5pt,% 右余白
  keepspaces=true,% 空白を表示する
  mathescape,% $ で囲った部分を数式として表示する ($ がソースコード中で使えなくなるので注意)
  % 手動強調表示の設定
  moredelim=[is][\itshape]{@/}{/@},
  moredelim=[is][\color{red}]{@r\{}{\}@},
  moredelim=[is][\color{blue}]{@b\{}{\}@},
  moredelim=[is][\color{DarkGreen}]{@g\{}{\}@},
}

\newcommand{\keyword}[1]{\mathbf{#1}}
\newcommand{\LET}{\keyword{let}}
\newcommand{\IF}{\keyword{if}}
\newcommand{\THEN}{\keyword{then}}
\newcommand{\ELSE}{\keyword{else}}
\newcommand{\FIX}{\keyword{fix}}
\newcommand{\CREATE}{\keyword{create}}
\newcommand{\AND}{\keyword{and}}
\newcommand{\IN}{\keyword{in}}
\newcommand{\TRUE}{\keyword{true}}
\newcommand{\WHILE}{\keyword{while}}
\newcommand{\DO}{\keyword{do}}
\newcommand{\DONE}{\keyword{done}}

%% ===============================================
%% 本文
%% ===============================================
\begin{document}
\frame[plain]{\titlepage}% タイトルページ

\begin{frame}
	\frametitle{背景:コンパイラの形式的検証の意義}
	\begin{center}
		\LARGE 
		\begin{columns}
			\begin{column}{0.8\textwidth}
				\begin{itembox}[c]{\Large コンパイラのバグ}
					\begin{itemize}
						\item 生成されるコードに影響
						\item デバッグが困難
					\end{itemize}
				\end{itembox}
			\end{column}
		\end{columns}
		$\Downarrow$\\
		形式的検証が有用
	\end{center}
\end{frame}

\begin{frame}
	\frametitle{背景:既存研究(1/2)}
	\begin{itemize}
		\item 高階関数がない
			\begin{itemize}
				\item CompCert [Leroy+ 2006]
					\begin{itemize}
						\item Cコンパイラの正当性の検証
					\end{itemize}
			\end{itemize}
		\item 入出力がない
			\begin{itemize}
				\item Lambda Tamer [Chlipala 2010]
					\begin{itemize}
						\item 非純粋高階関数型言語処理系の\\正当性の検証
					\end{itemize}
			\end{itemize}
	\end{itemize}
\end{frame}

\begin{frame}
	\frametitle{背景:既存研究(2/2)}
	\begin{itemize}
		\item 高階関数も入出力もある
			\begin{itemize}
				\item Pilsner [Neis+ 2015]
					\begin{itemize}
						\item ソース言語の束縛は名前で表現
						\item 小ステップ意味論
						\item 論理関係
					\end{itemize}
				\item CakeML [Tan+ 2016]
					\begin{itemize}
						\item ソース言語の束縛は名前で表現
						\item 中間言語の束縛はde Bruijnインデックスで表現
						\item クロックを用いた大ステップ意味論
					\end{itemize}
			\end{itemize}
	\end{itemize}
	\begin{flushright}
		\Large $\Rightarrow$後で手法を比較
	\end{flushright}
\end{frame}

\begin{frame}
	\frametitle{本研究の概要}
	\begin{itemize}
		\item 入出力を含む高階関数型言語処理系を検証
			\begin{itemize}
				\item 本研究ではK正規化パスのみを検証
				\item De Bruijnインデックスを採用
					\begin{itemize}
						\item 束縛の表現が簡潔
					\end{itemize}
				\item 余帰納的大ステップ意味論\mbox{[Leroy, Grall 2009]}を採用
					\begin{itemize}
						\item 無限に入出力が続く場合の検証が簡潔
					\end{itemize}
			\end{itemize}
	\end{itemize}
\end{frame}

\begin{frame}
	\frametitle{K正規化}
	\Large
	全ての部分式に名前を付ける(K正規形)変換

	\vfill

	例:
	\begin{columns}[c]
		\begin{column}{0.3\textwidth}
			\LARGE $a+b*c+d$
		\end{column}
		\begin{column}{0.1\textwidth}
			\LARGE $\Rightarrow$
		\end{column}
		\begin{column}{0.5\textwidth}
			\tikz[remember picture]{\node(let-insertion){
				\LARGE $\begin{array}{l}
					\alert<2>{\LET~x =} b * c~\alert<2>{\IN} \\
					\alert<2>{\LET~y =} a + x~\alert<2>{\IN} \\
					y + d
				\end{array}$};}
		\end{column}
	\end{columns}
	\pause
	\begin{tikzpicture}[remember picture, overlay]
		\node[rectangle callout, fill=red!80, white, callout absolute pointer={(let-insertion.north)}, above=of let-insertion]{束縛が導入される};
	\end{tikzpicture}
\end{frame}

\section*{アウトライン}

% 目次を表示させる(section を表示し、subsection は隠す)
%\begin{frame}
%  \frametitle{アウトライン}
%  \tableofcontents[sectionstyle=show,subsectionstyle=hide]
%\end{frame}

\section{対象言語}

\begin{frame}
	\frametitle{値呼びの型無し$\lambda$計算の拡張}
	{\large
	\[ \begin{array}{lcll}
		e & ::= & c & \mbox{定数(整数$\cdot$真偽値)} \\
				& | & e_1 \odot e_2 & \mbox{算術演算($+, -, \times, \leq$)} \\
				& | & \IF~e_1~\THEN~e_2~\ELSE~e_3 & \mbox{条件分岐} \\
				& | & \alert<2>{\LET~x=e_1~\IN~e_2} & \alert<2>{\mbox{変数定義}} \\
				& | & \alert<3>{x} & \alert<3>{\mbox{変数$\cdot$外部関数}} \\
				& | & \alert<4>{\FIX~f~x.~e} & \alert<4>{\mbox{再帰関数}} \\
				& | & e_1~e_2 & \mbox{関数呼び出し} \\
				& | & (e_1,~e_2) & \mbox{二つ組の作成} \\
				& | & \pi_i~e & \mbox{射影}(i \in \{0, 1\}) \\
	\end{array} \]}
	\Large
	\only<3>{\alert<3>{自由変数の適用}(例:$\texttt{print\_int 42}$)\\
		\alert<3>{で外部関数の呼び出し(入出力)を表現}}
\end{frame}

\section{束縛の表現}

\begin{frame}
	\frametitle{名前による表現の問題点}
	\begin{itemize}
		\item 項の等価性がCoqの等価性でなく\\$\alpha$等価性となる
			{\LARGE \[ \lambda x.\lambda y.~x =_\alpha \lambda a.\lambda b.~a \]}
		\item Captureを避ける必要がある
			{\LARGE \[\begin{array}{l}
				K(\alert{\lambda z}.~x+y+\alert{z}) \\
				\only<1>{\not=_\alpha}\only<2>{=_\alpha} \alert<2>{\lambda z.}~\only<1>{\alert{\LET~z=}}\only<2>{{\color{blue}\LET~a=}}x+y~\IN~\only<1>{z}\only<2>{{\color{blue}a}}+\alert{z}
			\end{array}\]}
	\end{itemize}
\end{frame}

\begin{frame}
	\frametitle{我々の解決策:de Bruijnインデックス}
	\begin{itemize}
		\item 内側から数えて何番目の束縛かで表現
			{\LARGE \[\begin{array}{lcl}
				\lambda x.~x~(\lambda y.~x~y) & \Rightarrow & \lambda.~0~(\lambda.~1~0)
			\end{array}\]}
		\item $\alpha$等価な式は文面上も同じ
			{\LARGE \[\begin{array}{lcl}
				\lambda x.\lambda y.~x & \Rightarrow & \lambda.\lambda.~1 \\
				\lambda a.\lambda b.~a & \Rightarrow & \lambda.\lambda.~1 \\
			\end{array}\]}
	\end{itemize}
\end{frame}

\begin{frame}[fragile]
	\frametitle{De Bruijnインデックスを用いたK正規化の実装}
	{\LARGE \[ \begin{array}{c}
		K(\alert{\lambda z.}~\texttt{f 42}+\alert{z})\\
		= \alert{\lambda z.}~\LET~a=\texttt{f 42}~\IN~a+\alert{z}\\
		\Downarrow\\
		K(\alert{\lambda.}~\texttt{f 42}+\alert{0})\\
		= \alert{\lambda.}~\LET~\texttt{f~42}~\IN~0+\alert{\only<1>{\uparrow^1 0}\only<2>{1}}
	\end{array}\]}
	\vfill
	\begin{itemize}
		\item \alert{インデックスをずらす必要がある}\\
		{\large $\Rightarrow$シフト(インデックスをずらす操作)を用いる}
	\end{itemize}
\end{frame}

\begin{frame}
	\Large
	ただし外部関数は名前で表現
	\vfill
	{\LARGE \[\begin{array}{c}
		(\FIX~f~\alert{x}.~\texttt{print\_int}~\alert{x})~\texttt{42}\\
		\Downarrow\\
		(\alert{\FIX.}~\texttt{print\_int}~\alert{1})~\texttt{42}
	\end{array}\]}
\end{frame}

\subsection{他手法との比較}

\begin{frame}
	\frametitle{比較:局所名前無し表現 [Gordon 1993]}
	\Large
	\begin{itemize}
		\item 自由変数は名前で表現
		\item 束縛変数はde Bruijnインデックスで表現\\
			\begin{itemize}
				\item Well-formedである証明が必要
			\end{itemize}
			$K(e_1~e_2) = \LET~ K(e_1)~\IN~\LET~K(e_2)~\IN~1~0$
			\begin{itemize}
				\item シフトが不要になる
			\end{itemize}
			$K(\alert<4>{\lambda.~e}) = \lambda.~\alert<3>{[x\mapsto 0]}(K(\alert<4>{\alert<2>{[0\mapsto x]}e}))$
			\begin{itemize}
				\item $[0\mapsto x]e$が$\lambda.~e$の部分式にならない
			\end{itemize}
	\end{itemize}
	\pause
\end{frame}

\begin{frame}
	\frametitle{比較:PHOAS [Chlipala 2008]}
	\LARGE ホスト言語の束縛で対象言語の束縛を表現\\
	例:$\begin{array}{lcl}
		\lambda x.~x & \Rightarrow & \texttt{Abs(fun x=>Var x)}
	\end{array}$
	\begin{itemize}
		\item 名前の付け替えや代入をホスト言語に任せられる
		\item Parametricityの証明が必要
	\end{itemize}
\end{frame}

\section{無限の入出力}

\begin{frame}
	\frametitle{有限の入出力の場合}
	\LARGE
	\fbox{$e \Downarrow v~/~t$}
	\begin{prooftree}
		\AxiomC{$e_1 \Downarrow \FIX .~e~/~t_1$}
		\AxiomC{$e_2 \Downarrow v_2~/~t_2$}
		\noLine
		\BinaryInfC{$[0\mapsto \FIX.~e,~1\mapsto v_2]e \Downarrow v~/~t_3$}
		\UnaryInfC{$e_1~e_2 \Downarrow v~/~t_1,~t_2,~t_3$}
	\end{prooftree}
	\begin{prooftree}
		\AxiomC{$e_1 \Downarrow x~/~t_1$}
		\AxiomC{$e_2 \Downarrow n~/~t_2$}
		\BinaryInfC{\tikz[remember picture]{\node(nondeterminism){$e_1~e_2 \Downarrow \alert<2>{m}~/~t_1,~t_2~,~x(n)=\alert<2>{m}$};}}
	\end{prooftree}
	\pause
	\Large
	\begin{tikzpicture}[remember picture, overlay]
		\node[above right = 2mm and 3.7cm of nondeterminism.south](nondeterminism-){};
		\node[rectangle callout, fill=red!80, white, callout absolute pointer={(nondeterminism-.south)}, below=1mm of nondeterminism-.south]{mは非決定的};
	\end{tikzpicture}
\end{frame}

\begin{frame}
	\frametitle{大ステップ意味論の問題点}
	\LARGE
	\begin{center}
		停止しない式を区別不能
	\end{center}
	\vfill
	\begin{columns}
		\begin{column}{0.5\textwidth}
			\begin{center}
				{\LARGE 有限回の入出力}
				\normalsize \[(\FIX~f~x.~f~x)~(\texttt{print\_int}~\texttt{42})\]
			\end{center}
		\end{column}
		\vline{}
		\begin{column}{0.5\textwidth}
			\begin{center}
				{\LARGE 無限回の入出力}
				\normalsize \[(\FIX~f~x.~f~(\texttt{print\_int}~\texttt{42}))~\texttt{0}\]
			\end{center}
		\end{column}
	\end{columns}
\end{frame}

\begin{frame}
	\frametitle{余帰納的大ステップ意味論 [Leroy+ 2009] (1/2)}
	\LARGE
	\fbox{$e \Uparrow /~t$}
	\begin{prooftree}
		\AxiomC{$e_1 \Uparrow /~t_1$}
		\doubleLine
		\UnaryInfC{$e_1~e_2 \Uparrow /~t_1$}
	\end{prooftree}
	\vfill
	\begin{prooftree}
		\AxiomC{$e_1 \Downarrow \FIX.~e~/~t_1$}
		\AxiomC{$e_2 \Uparrow /~ t_2$}
		\doubleLine
		\BinaryInfC{$e_1~e_2 \Uparrow /~t_1~t_2$}
	\end{prooftree}
	\vfill
	\begin{prooftree}
		\AxiomC{$e_1 \Downarrow \FIX.~e~/~t_1$}
		\AxiomC{$e_2 \Downarrow v_2~/~t_2$}
		\noLine
		\BinaryInfC{$[0 \mapsto \FIX~e,~1\mapsto v_2]e \Uparrow / ~t_3$}
		\doubleLine
		\UnaryInfC{$e_1~e_2 \Uparrow/~t_1~t_2~t_3$}
	\end{prooftree}
\end{frame}

\begin{frame}
	\frametitle{余帰納的大ステップ意味論 [Leroy+ 2009] (3/3)}
	\LARGE 
	\begin{center}
		停止しない式同士も区別可能
	\end{center}
	\vfill
	\begin{columns}
		\begin{column}{0.5\textwidth}
			\begin{center}
				\LARGE 有限回の入出力
			\end{center}
			\normalsize \[\begin{array}{l}
				(\FIX~f~x.~f~x)~(\texttt{print\_int 42}) \\
					\Uparrow /~\texttt{print\_int}(42)=0
					\\
					\\
					\\
					\\
			\end{array}\]
		\end{column}
		\vline{}
		\begin{column}{0.5\textwidth}
			\begin{center}
				\LARGE 無限回の入出力
			\end{center}
			\normalsize \[\begin{array}{l}
				(\FIX~f~x.~f~(\texttt{print\_int}~\texttt{42}))~\texttt{10} \\
					\Uparrow /~\texttt{print\_int}(42)=0, \\
					\texttt{print\_int}(42)=0, \\
					\texttt{print\_int}(42)=0, \\
					\cdots
			\end{array}\]
		\end{column}
	\end{columns}
\end{frame}

\begin{frame}
	\frametitle{評価関係の導出例}
	\begin{prooftree}
				\AxiomC{}
				\UnaryInfC{$\texttt{p}\Downarrow\texttt{p}/$}

				\AxiomC{}
				\UnaryInfC{$\texttt{42}\Downarrow\texttt{42}/$}

			\BinaryInfC{$\texttt{p 42}\Downarrow 0/\texttt{p(42)=0}$}

			\AxiomC{$\vdots$}
			\doubleLine
			\UnaryInfC{$(\FIX.0(\texttt{p 42}))(\texttt{p 42})\Uparrow/\texttt{p(42)=0}\cdots$}

		\doubleLine
		\BinaryInfC{$(\FIX.0(\texttt{p 42}))(\texttt{p 42})\Uparrow/\texttt{p(42)=0,p(42)=0}\cdots$}
	\end{prooftree}
\end{frame}

\subsection{他手法との比較}

\begin{frame}
	\frametitle{関連研究:ラベル付き小ステップ意味論}
	\LARGE 
	{\Large \[\begin{array}{rl}
		& \texttt{read\_int}(\texttt{0}) + 34\\
		\xrightarrow{\texttt{read\_int(0)}=12} & 12 + 34\\
		\xrightarrow{\tau} & 45
	\end{array}\]}
	\vfill
	$\tau$遷移の取り扱いが煩雑
	\begin{itemize}
		\item 弱双模倣やup-to reductionが必要
	\end{itemize}
\end{frame}

\section{正当性の検証}

\begin{frame}
	\frametitle{K正規化の正当性}
	{\LARGE
	\begin{theorem}\label{theorem:evalto-correctness}
		$e\Downarrow v~/~t$ならば,$K(e)\Downarrow v'~/~t$かつ\\
		$v$が整数または真偽値ならば$v=v'$
	\end{theorem}
	\begin{theorem}
		$e\Uparrow/~t$ならば$K(e)\Uparrow/~t$
	\end{theorem}}
	{\large 定理2は余帰納法で証明できる}
\end{frame}

\begin{frame}
	\frametitle{定理1の証明の問題点}
	\LARGE 定理1を証明するには\\帰納法の仮定の強化が必要
	\vfill
	\begin{hypothesis}
		{\Large $e\Downarrow v~/~t$がならば$K(e)\Downarrow K(v)~/~t$}
	\end{hypothesis}
	一見成り立ちそうに見える
\end{frame}

\begin{frame}
	\frametitle{仮説1の問題点}
	\LARGE 複合的な値(組)があるため成り立たない

	\vfill
	\renewcommand{\thehypothesis}{1}
	\begin{hypothesis}
		{\Large $e\Downarrow v~/~t$がならば$K(e)\Downarrow K(v)~/~t$}
	\end{hypothesis}
	\Large
	$v=(v_1,~v_2)$のとき\\
	\alert{\tikz[remember picture]{\node(Kv){$K(v)=\LET~K(v_1)~\IN~\LET~\uparrow^1 K(v_2)~\IN~(1,0)$};}}
	従って$K(e)\not\Downarrow K(v)~/~t$
	\begin{tikzpicture}[remember picture, overlay]
		\node[right = 3cm of Kv.south](Kv-){};
		\node[rectangle callout, fill=red!80, white, callout absolute pointer={(Kv-)}, below = 2mmof Kv-]{値ですらない};
	\end{tikzpicture}
\end{frame}

\begin{frame}
	\frametitle{解決策}
	\large
	\begin{block}{補題}
		$[0\mapsto v_0,\cdots , n-1\mapsto v_{n-1}]e\Downarrow v~/~t$かつ\\
		$\forall i\in \{0,\cdots,n-1\}.~v_i \rhd v_i'$がならば,
		\[[0\mapsto v_0',\cdots , n-1\mapsto v_{n-1}']K(e)\Downarrow v'~/~t\]及び
		$v\rhd v'$となる$v'$が存在
	\end{block}
	\only<1>{\begin{itemize}
		\item $[0\mapsto v_0,\cdots , n-1\mapsto v_{n-1}]e\Downarrow v~/~t$\\
			の導出についての帰納法で証明
			\begin{itemize}
				\item 関数適用や$\LET$の評価で代入が起こる
			\end{itemize}
	\end{itemize}}
	\pause
	\begin{prooftree}
		\AxiomC{$\forall i\in \{2,\cdots,n+1\}.~v_i \rhd v_i'$}
		\UnaryInfC{$\begin{array}{c}
			\FIX.~[2\mapsto v_2,~\cdots , n+1\mapsto v_{n+1}]e\\
			\rhd~\FIX.~[2\mapsto v_2',~\cdots , n+1\mapsto v_{n+1}']K(e) \\
		\end{array}$}
	\end{prooftree}
	\begin{prooftree}
		\AxiomC{$v_1 \rhd v_1'$}
		\AxiomC{$v_2 \rhd v_2'$}
		\BinaryInfC{$(v_1,~v_2) \rhd (v_1',~v_2')$}
	\end{prooftree}
\end{frame}

\subsection{他手法との比較}

\begin{frame}
	\frametitle{比較:論理関係 (Logical Relation)}
	\begin{itemize}
		\item プログラム等価性に用いられる証明手法
		\begin{itemize}
			\item 型無しの言語に適用しづらい
			\item 再帰関数の扱いが面倒
				\begin{itemize}
					\item Unwindingに関する補題が必要
				\end{itemize}
		\end{itemize}
	\end{itemize}
\end{frame}

\section{結論}

\begin{frame}
	\frametitle{まとめ}
	\LARGE 入出力を含む高階関数型言語の\\
	K正規化をCoqで検証
	\begin{itemize}
		\item 無限に入出力が続く場合も検証
		\item De Bruijnインデックスや\\
			余帰納的大ステップ意味論の採用で証明を簡潔に
	\end{itemize}
\end{frame}

\begin{frame}
	\frametitle{今後の課題}
	\begin{itemize}
		\item Arity nの式
			\begin{itemize}
				\item 複数引数の関数呼び出し
				\item 相互再帰
				\item 組
					\begin{itemize}
						\item Coqでの取り扱いが煩雑
					\end{itemize}
			\end{itemize}
		\item 入出力以外の副作用
			\begin{itemize}
				\item 例外
				\item 配列操作
			\end{itemize}
	\end{itemize}
\end{frame}

\end{document}
